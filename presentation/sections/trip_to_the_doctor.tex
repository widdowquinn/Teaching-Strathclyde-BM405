
% SUBSECTION: a short aside
\subsection{A visit to the doctor}

\begin{frame}
  \frametitle{Why Performance Metrics Matter}
  \begin{itemize}
    \item<1-> You go for a checkup, and are tested for disease $X$
    \item<1-> The test has \textbf{$\text{sensitivity}=0.95$} (predicts disease where there is disease)
    \item<1-> The test has \textbf{$\text{FPR}=0.01$} (predicts disease where there is no disease)
    \item<2-> Your test is \emph{positive}
    \item<2-> \textbf{What is the probability that you have disease $X$?}
    \begin{itemize}
      \item \textbf{0.01, 0.05, 0.50, 0.95, 0.99?}
    \end{itemize}
  \end{itemize} 
\end{frame}

\begin{frame}
  \frametitle{Why Performance Metrics Matter}
  \begin{itemize}
    \item<1-> What is the probability that you have disease $X$?
    \item<1-> \textbf{Unless you know the \emph{baseline occurrence} of disease $X$, you cannot determine this.}
    \item<2-> Baseline occurrence: $f_X$
    \begin{itemize}
      \item $f_X = 0.01 \implies P(\text{disease}|\text{+ve}) = 0.490 \approx 0.5$
      \item $f_X = 0.8 \implies P(\text{disease}|\text{+ve}) = 0.997 \approx 1.0$         
    \end{itemize}
  \end{itemize} 
\end{frame}

\begin{frame}
  \frametitle{Why Performance Metrics Matter}
  \begin{itemize}
    \item<1-> Imagine a predictor for protein functional class
    \item<1-> Predictor has has $\text{sensitivity}=0.95$, $\text{FPR}=0.01$
    \item<1-> You run the predictor on 20,000 proteins in an organism
    \item<2-> We estimate $\approx$ 200 members in protein complement, so $f_X=0.01$
    \begin{itemize}
      \item $f_X = 0.01 \implies P(\text{class}|\text{+ve}) = 0.490 \approx 0.5$
    \end{itemize}
  \end{itemize} 
\end{frame}

\begin{frame}
  \frametitle{Bayes' Theorem}
  \begin{itemize}
    \item May seem counter-intuitive: 95\% sensitivity, 99\% specificity $\implies$ 50\% chance of any prediction being incorrect
    \item Probability given by Bayes' Theorem
    \begin{itemize}
      \item $P(X|+) =  \frac{P(+|X) P(X)}{P(+|X) P(X) + P(+|\bar{X}) P(\bar{X})}$
    \end{itemize}
    \item This step commonly overlooked in the literature (confirmation bias?)
    \item e.g. in paper describing novel TTSS predictor: \\
      ``The surprisingly high number of (false) positives in genomes without TTSS exceeds the expected false positive rate"
  \end{itemize} 
\end{frame}

\begin{frame}
  \frametitle{Interpreting Performance Metrics}
  \begin{itemize}
    \item<1-> Use Bayes' Theorem!
    \item<1-> Predictions identify groups, not individual members of the group. e.g.
    \begin{itemize}
      \item Test for airport smugglers has $P(\text{smuggler}|+) = 0.9$
      \item Test gives 100 positives
    \end{itemize}
    \item<1-> Which specific individuals are truly smugglers?
    \item<2-> The test \emph{does not} allow you to determine this - you need more evidence for each individual
    \item<2->  Same principle applies to all other tests, (including protein functional class prediction) - you should not `cherry-pick' for publication without other evidence
  \end{itemize} 
\end{frame}

