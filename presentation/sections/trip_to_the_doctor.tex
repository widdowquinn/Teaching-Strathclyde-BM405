%% trip_to_the_doctor.tex
%% Author: Leighton Pritchard
%% Copyright: James Hutton Institute
%% A brief example to demonstrate the importance of accounting for multiple-testing

% SUBSECTION: a short aside
\subsection{A visit to the doctor}

\begin{frame}
  \frametitle{A wee trip to the doctor}
  \begin{itemize}
    \item<1-> You go for a checkup, and are tested for disease $X$
    \item<1-> The test has \textbf{$\text{sensitivity}=0.95$} (predicts disease where there is disease)
    \item<1-> The test has \textbf{$\text{FPR}=0.01$} (predicts disease where there is no disease)
    \item<2-> Your test is \emph{positive}
    \item<2-> \textbf{What is the probability that you have disease $X$?}
    \begin{itemize}
      \item \textbf{0.01, 0.05, 0.50, 0.95, 0.99?}
    \end{itemize}
    \item<2-> (Audience Participation!)
  \end{itemize} 
\end{frame}

\begin{frame}
  \frametitle{A wee trip to the doctor}
  \begin{itemize}
    \item<1-> What is the probability that you have disease $X$?
    \item<1-> \textbf{Unless you know the \emph{baseline occurrence} of disease $X$, you cannot determine this.}
    \item<2-> Baseline occurrence: $f_X$
    \begin{itemize}
      \item $f_X = 0.01 \implies P(\text{disease}|\text{+ve}) = 0.490 \approx 0.5$
      \item $f_X = 0.8 \implies P(\text{disease}|\text{+ve}) = 0.997 \approx 1.0$         
    \end{itemize}
  \end{itemize} 
\end{frame}



